%!TEX TS-program = xelatex
%!TEX encoding = UTF-8 Unicode
% Awesome CV LaTeX Template for CV/Resume
%
% This template has been downloaded from:
% https://github.com/posquit0/Awesome-CV
%
% Author:
% Claud D. Park <posquit0.bj@gmail.com>
% http://www.posquit0.com
%
% Template license:
% CC BY-SA 4.0 (https://creativecommons.org/licenses/by-sa/4.0/)
%


%-------------------------------------------------------------------------------
% CONFIGURATIONS
%-------------------------------------------------------------------------------
% A4 paper size by default, use 'letterpaper' for US letter
\documentclass[11pt, a4paper]{awesome-cv}

% Configure page margins with geometry
\geometry{left=1.4cm, top=.8cm, right=1.4cm, bottom=1.8cm, footskip=.5cm}

% Specify the location of the included fonts
\fontdir[fonts/]

% Color for highlights
% Awesome Colors: awesome-emerald, awesome-skyblue, awesome-red, awesome-pink, awesome-orange
%                 awesome-nephritis, awesome-concrete, awesome-darknight
\definecolor{pkured}{RGB}{140, 0, 0}
\colorlet{awesome}{pkured}
% Uncomment if you would like to specify your own color
% \definecolor{awesome}{HTML}{CA63A8}

% Colors for text
% Uncomment if you would like to specify your own color
% \definecolor{darktext}{HTML}{414141}
% \definecolor{text}{HTML}{333333}
% \definecolor{graytext}{HTML}{5D5D5D}
% \definecolor{lighttext}{HTML}{999999}

% Set false if you don't want to highlight section with awesome color
\setbool{acvSectionColorHighlight}{true}

% If you would like to change the social information separator from a pipe (|) to something else
\renewcommand{\acvHeaderSocialSep}{\quad\textbar\quad}


%-------------------------------------------------------------------------------
%	PERSONAL INFORMATION
%	Comment any of the lines below if they are not required
%-------------------------------------------------------------------------------
% Available options: circle|rectangle,edge/noedge,left/right
% \photo{./examples/profile.png}
\name{Kenuo}{Xu}
\position{Ph.D. Student{\enskip\cdotp\enskip}Peking University}
\address{Room 513S, Science Building No.5, 5 Yiheyuan Road, Beijing, 100871}


\email{kenuo.xu@pku.edu.cn}
\homepage{https://witty-me.github.io/}
\github{Witty-me}
\googlescholar{loX-G0gAAAAJ}{Google Scholar}
% \linkedin{posquit0}
% \gitlab{gitlab-id}
% \stackoverflow{SO-id}{SO-name}
% \twitter{@twit}
% \skype{skype-id}
% \reddit{reddit-id}
% \medium{madium-id}
% \googlescholar{googlescholar-id}{name-to-display}
%% \firstname and \lastname will be used
% \googlescholar{googlescholar-id}{}
% \extrainfo{extra informations}

% \quote{``Be the change that you want to see in the world."}


%-------------------------------------------------------------------------------
\begin{document}

% Print the header with above personal informations
% Give optional argument to change alignment(C: center, L: left, R: right)
\makecvheader

% Print the footer with 3 arguments(<left>, <center>, <right>)
% Leave any of these blank if they are not needed
\makecvfooter
  {\today}
  {Kenuo Xu~~~·~~~Curriculum Vitae}
  {\thepage}


%-------------------------------------------------------------------------------
%	CV/RESUME CONTENT
%-------------------------------------------------------------------------------

\cvsection{Education}
\begin{cventries}
	
	%---------------------------------------------------------
	\cventry
	{Ph.D. in Computer Science}
	{Peking University}
	{Beijing, China}
	{Sep. 2020 - Jun. 2025 (Expected)}
	{
		\begin{cvitems} % Description(s) bullet points
			\item {In the Software-hardware Orchestrated ARchitecture (SOAR) group; advisor: Prof. Chenren Xu}
			\item {Design a visible light backscatter communication system that supports concurrent transmission for low latency purpose.}
			\item {Design a visible light communication system with spike cameras as receivers to achieve high data rate and dynamic range.}
			\item {Design a liquid-crystal fiducial marker system using LiDAR as receivers for extended reading range and higher ranging accuracy.}
			\item {See publications for more research.}
		\end{cvitems}
	}

	\cventry
	{B.Sc. in Computer Science} % Degree
	{Peking University} % Institution
	{Beijing, China} % Location
	{Sep. 2016 - Jun. 2020} % Date(s)
	{
		\begin{cvitems} % Description(s) bullet points
			\item {Graduated with Excellent Graduate Award}
		\end{cvitems}
	}

	%---------------------------------------------------------
\end{cventries}

\cvsection{Employment}
\begin{cventries}
	
	%---------------------------------------------------------
	\cventry
	{Research Intern} % Degree
	{Microsoft Research Asia} % Institution
	{Shanghai, China} % Location
	{Dec. 2022 - Sep. 2023} % Date(s)
	{
		\begin{cvitems} % Description(s) bullet points
			\item {In the Shanghai Wireless Group; mentor: Prof. Lili Qiu}
			\item {Worked on large language models plus computer networking.}
		\end{cvitems}
	}
	
	%---------------------------------------------------------
\end{cventries}


\cvsection{Publications}
\begin{cventries}
	
	%---------------------------------------------------------
	\cventry
	{\textbf{Kenuo Xu}, Bo Liang, Jingyu Li, Chenren Xu}
	{RetroLiDAR: A Liquid-crystal Fiducial Marker System for High-fidelity Spatial Computing}
	{In Submission}
	{2024}
	{
		\begin{cvitems} % Description(s) bullet points
			\item {A long-range high-ranging-accuracy fiducial marker system for robotics and virtual reality.}
		\end{cvitems}
	}
		
	\cventry
	{Zhiyuan He, Aashish Gottipati, Lili Qiu, Fanscis Y. Yan, Xufang Luo, \textbf{Kenuo Xu}, Yuqing Yang}
	{LLM-ABR: Designing Adaptive Bitrate Algorithms via Large Language Models}
	{In Submission}
	{2023}
	{
		\begin{cvitems} % Description(s) bullet points
			\item {Using LLMs to design algorithms tailored for computer networks.}
		\end{cvitems}
	}
		
	\cventry
	{Chenren Xu, \textbf{Kenuo Xu}, Lilei Feng, Bo Liang}
	{RetroV2X: A New V2X Paradigm with Visible Light Backscatter Networking}
	{Fundamental Research}
	{2023}
	{
		\begin{cvitems} % Description(s) bullet points
			\item {A practical vehicle-to-everything communication system with visible light.}
		\end{cvitems}
	}
	
	\cventry
	{\textbf{Kenuo Xu}, Kexing Zhou, Chengxuan Zhu, Shanghang Zhang, Boxin Shi, Xiaoqiang Li, Tiejun Huang, Chenren Xu}
	{When Visible Light (Backscatter) Communication Meets Neuromorphic Cameras in V2X}
	{ACM HotMobile}
	{2023}
	{
		\begin{cvitems} % Description(s) bullet points
			\item {When VLC meets neuromorphic cameras: a spike cameras as VLC receiver achieves 4.8 kbps data rate and different mobile scenarios.}
		\end{cvitems}
	}

	\cventry
	{\textbf{Kenuo Xu}, Chen Gong, Bo Liang, Yue Wu, Boya Di, Lingyang Song, Chenren Xu}
	{Low-Latency Visible Light Backscatter Networking with RetroMUMIMO}
	{ACM SenSys}
	{2022}
	{
		\begin{cvitems} % Description(s) bullet points
			\item {Enables 8 concurrent VLBC links and reduces networking latency by 92.0\%.}
		\end{cvitems}
	}

	\cventry
	{Chenren Xu, Purui Wang, Tuochao Chen, Yue Wu, \textbf{Kenuo Xu}, Xieyang Xu, Yang Shen, Junrui Yang, Guojun Chen, Guobin Shen}
	{VLID: Visible Light Backscatter System for Battery-free Internet-of-Things}
	{IEEE/ACM Transactions on Networking}
	{Accepted}
	{
		\begin{cvitems} % Description(s) bullet points
			\item {An end-to-end VLBC solution for battery-free IoT networking.}
		\end{cvitems}
	}
	
	\cventry
	{Purui Wang, Lilei Feng, Guojun Chen, Chenren Xu, Yue Wu, \textbf{Kenuo Xu}, Guobin Shen, Kuntai Du, Gang Huang, Xuanzhe Liu}
	{Renovating road signs for infrastructure-to-vehicle networking: a visible light backscatter communication and networking approach}
	{ACM MobiCom}
	{2020}
	{
		\begin{cvitems} % Description(s) bullet points
			\item {Enhance the reliability of autonomous driving with reflective roadsigns that conveys dynamic additional information.}
		\end{cvitems}
	}
	
	\cventry
	{Yue Wu, Purui Wang, \textbf{Kenuo Xu}, Lilei Feng, Chenren Xu}
	{Turboboosting Visible Light Backscatter Communication}
	{ACM SIGCOMM}
	{2020}
	{
		\begin{cvitems} % Description(s) bullet points
			\item {Improve the data rate of VLBC by 8x (prototype) and 32x (simulation) with advanced modulation schemes.}
		\end{cvitems}
	}
	
	
	%---------------------------------------------------------
\end{cventries}

\cvsection{Honors \& Awards}

\begin{cvhonors}
	
	%---------------------------------------------------------
	\cvhonor
	{Merit Student} % Award
	{Peking University} % Event
	{Beijing, China} % Location
	{2022} % Date(s)
	
	%---------------------------------------------------------
	\cvhonor
	{First Prize} % Award
	{Competition of Future Network Technology Innovation} % Event
	{Nanjing, China} % Location
	{2021} % Date(s)
	
	%---------------------------------------------------------
	\cvhonor
	{Excellent Graduate} % Award
	{Peking University} % Event
	{Beijing, China} % Location
	{2020} % Date(s)
	
	%---------------------------------------------------------
	\cvhonor
	{Houston BAA Scholarship} % Award
	{Peking University} % Event
	{Beijing, China} % Location
	{2019} % Date(s)
	
	%---------------------------------------------------------
	\cvhonor
	{Merit Student} % Award
	{Peking University} % Event
	{Beijing, China} % Location
	{2019} % Date(s)

	%---------------------------------------------------------
\end{cvhonors}

\cvsection{Activities}
\begin{cventries}
	
	%---------------------------------------------------------
	\cventry
	{Computer Networks (Honor Track)}
	{Teaching Assistant}
	{Peking University}
	{Fall 2019, 2020, 2021(Light), 2022(Light)}
	{
		\begin{cvitems} % Description(s) bullet points
			\item {Organizing the course and answering questions.}
			\item {Giving assignments, tutorials, and grading of labs.}
			\item {Designing quizzes and grading students' responses.}
			\item {Mentoring course research projects (light).}
		\end{cvitems}
	}
	
	\cventry
	{Proceedings of the ACM on Interactive, Mobile, Wearable and Ubiquitous Technologies (IMWUT)} % Degree
	{Journal Reviewer} % Institution
	{} % Location
	{2021} % Date(s)
	{}	
	
	%---------------------------------------------------------
\end{cventries}

%\input{cv/education.tex}
%\input{cv/skills.tex}
%\input{cv/experience.tex}
%\input{cv/extracurricular.tex}
%\input{cv/honors.tex}
%\input{cv/presentation.tex}
%\input{cv/writing.tex}
%\input{cv/committees.tex}


%-------------------------------------------------------------------------------
\end{document}
